\documentclass[12pt,a4paper,openright,twoside]{book}
\usepackage[utf8]{inputenc}
\usepackage{disi-thesis}
\usepackage{code-lstlistings}
\usepackage{notes}
\usepackage{shortcuts}
\usepackage{acronym}

\school{\unibo}
\programme{Corso di Laurea in Ingegneria e Scienze Informatiche}
\title{Slime-Mold Aggregation}
\author{Lorenzo Tosi}
\date{\today}
\subject{Programmazione Ad Oggetti}
\supervisor{Prof. Mirko Viroli}
\cosupervisor{Dott. Gianluca Aguzzi}
\session{IV}
\academicyear{2022-2023}

% Definition of acronyms
\acrodef{IoT}{Internet of Thing}
\acrodef{vm}[VM]{Virtual Machine}


\mainlinespacing{1.241} % line spacing in mainmatter, comment to default (1)

\begin{document}

\frontmatter\frontispiece

\begin{abstract}	
Max 2000 characters, strict.
\end{abstract}

\begin{dedication} % this is optional
Optional. Max a few lines.
\end{dedication}

\begin{acknowledgements} % this is optional
Optional. Max 1 page.
\end{acknowledgements}

%----------------------------------------------------------------------------------------
\tableofcontents   
\listoffigures     % (optional) comment if empty
\lstlistoflistings % (optional) comment if empty
%----------------------------------------------------------------------------------------

\mainmatter

%----------------------------------------------------------------------------------------
\chapter{Introduzione}
\label{chap:introduzione}
%----------------------------------------------------------------------------------------

Nel vasto campo della ricerca scientifica, i comportamenti complessi emergenti sono oggetto di 
crescente interesse e studio. Questi fenomeni rappresentano il manifestarsi di 
comportamenti collettivi che sorgono dall'interazione dinamica e non lineare di 
molteplici componenti di un sistema, difficilmente anticipabili in base alle leggi che manovrano le 
singole parti.

In natura questi comportamenti sono ubiqui e possono essere osservati in tantissimi ambiti, dal regno 
animale, dove possiamo trovare per esempio la forma e il comportamento di uno stormo di uccelli o di un branco di pesci, 
al comportamento dell'uomo osservato durante il traffico nelle città, dal mercato della borsa valori al gioco del poker.

Un esempio significativo è quello osservato in biologia in una colonia di formiche. Nonostante manchi
una struttura centralizzata e le formiche seguono regole di comportamento semplici e locali, 
l'interazione tra di esse dà origine ad una "comunità" e di conseguenza a modelli complessi di ricerca del cibo, 
di costruzione di nidi e di difesa del territorio.
Ogni formica reagisce a stimoli, sotto forme di tracce chimiche, che provengono da altre formiche
e a sua volta influenza i suoi simili lasciando dietro una traccia chimica.
Questo fenomeno è simile ad altre strutture emergenti presenti in natura e riscontrate sia negli "insetti sociali",
ovvero insetti che formano colonie con mansioni diversificate, o in generale in animali che vivono in gruppo 
(pesci, tartarughe, mandrie di mammiferi,...) e sono tutti basati principalmente su feromoni o odori chimici.

%Perché è interessante la simulazione di questi comportamenti?
La simulazione di questi fenomeni è estremamente importante per diversi motivi:
\begin{itemize}
    \item Comprenderne la complessità: I fenomeni complessi sono caratterizzati da interazioni dinamiche tra
    i suoi comonenti che spesso sono imprevedibili. La simulazione è una risorsa chiave per esplorare, studiare e comprendere 
    moltissimi aspetti di queste dinamiche e permette di osservare le interazioni dei diversi elementi in infiniti modi.
    \item Predire il comportamento del sistema: la simulazione può essere eseguita per cercare di prevedere e comprendere il comportamento
    futuro di un sistema emergente in modo tale da poter prendere delle decisioni informate.
\end{itemize}

L'obiettivo di questa tesi è esplorare il fenomeno dell'aggregazione di questi organismi, sviluppando un sistema software che 
si interfacci ed utilizzi a pieno tutti gli elementi chiave del simulatore Alchemist. Quest'ultimo, infatti, permette di riprodurre eventi appartenenti 
a domini estremamente differenti tra loro, come simulazioni chimiche o il comportamento di pedoni in diverse situazioni.


% \sidenote{Add sidenotes in this way. They are named after the author of the thesis}
%You can use acronyms that your defined previously,
%such as \ac{IoT}.
%
%If you use acronyms twice,
%they will be written in full only once
%(indeed, you can mention the \ac{IoT} now without it being fully explained).
%
%In some cases, you may need a plural form of the acronym.
%
%For instance,
%that you are discussing \acp{vm},
%you may need both \ac{vm} and \acp{vm}.

\paragraph{Structure of the Thesis}

\note{At the end, describe the structure of the paper}

\chapter{Contesto}
In questo capitolo vengono spiegate le tecnologie adottate 
%I suggest referencing stuff as follows: \cref{fig:random-image} or \Cref{fig:random-image}

\section{Alchemist}
Alchemist \cite{Pianini_2013} è un simulatore DES (Discrete Event System) che estende il modello computazionale 
base delle reazioni chimiche in modo tale da favorirne l’applicabilità a situazioni complesse,
pur mantenendo elevate prestazioni. In particolare, Alchemist si fonda su una versione ottimizzata 
dell’algoritmo di Gillespie\cite{gillespie1977exact} chiamata Next Reaction Method\cite{gibson2000efficient}, esteso in modo tale da poter lavorare 
con un ambiente mobile e dinamico dove sia possibile aggiungere o rimuovere reazioni, dati e
connessioni topologiche. Le applicazioni già implementate sono varie e comprendono, ad esempio, 
simulazioni di reazioni biochimiche e movimento di pedoni. Il punto di forza del sistema è il 
meta-modello estremamente astratto, la cui effettiva realizzazione è demandata alle "incarnazioni",
le quali rappresentano l’implementazione vera e propria delle diverse categorie di simulazioni 
eseguibili all’interno. Attualmente troviamo 4 incarnazioni: 
\begin{itemize}
    \item Protelis
    \item SAPERE
    \item Biochemistry
    \item Scafi
\end{itemize}
\subsection{Il meta-modello}

Come accennato in precedenza, il meta-modello è uno dei punti di forza maggiori del simulatore. 
Per meta-modello si intende un tipo di paradigma che descrive la struttura, le regole e le relazioni
che i modelli di dati devono seguire all’interno di un sistema. Rappresenta in modo astratto i 
concetti e le relazioni all’interno del dominio di interesse e stabilisce i vincoli e le convenzioni
che tutti i modelli devono usare. Dunque, tutte le incarnazioni presenti presentano le stesse entità
base. Poichè Alchemist è sviluppato partendo da un’ispirazione orientata sulla chimica/biochimica,
le entità presentano nomi riconducibili a quei mondi. Infatti troviamo:
\begin{itemize}
    \item \textbf{Molecole} (molecule): il nome di un dato, concettualmente può essere interpretato come il nome di una variabile.
    \item \textbf{Concentrazioni} (concentration): il valore associato alla molecola.
    \item \textbf{Nodi} (node): un “contenitore” di molecole e reazioni.
    \item \textbf{Ambiente} (environment): l’astrazione dello spazio; è un “contenitore” di nodi e svolge i seguenti compiti:
    \begin{itemize}
        \item Restituire la posizione dei nodi.
        \item Restituire la distanza tra due nodi.
        \item Supportare il movimento dei nodi, se presente.
    \end{itemize}
    \item \textbf{Regola di collegamento} (linking rule): una funzione relativa allo stato corrente dell’ambiente che associa ad ogni nodo un vicinato.
    \item \textbf{Vicinato} (neighborhood): un entità composta da un nodo centrale e un insieme di nodi vicini.
    \item \textbf{Reazione} (reaction): un qualsiasi evento che provoca un cambiamento dello stato dell’ambiente. È definita da una lista di condizioni, una o più lista di azioni e una distribuzione temporale. La frequenza con la quale avviene una reazione dipende da:
    \begin{itemize}
        \item Un parametro statico “rate”.
        \item Il valore di ogni condizione.
        \item Una “rate equation”, ovvero una equazione che combina il parametro statico (rate) con i valori delle condizioni, restituendo un “instantaneous rate”.
        \item Una distribuzione temporale.
    \end{itemize}
    \item \textbf{Condizione} (condition): una funzione che, dato lo stato attuale del’ambiente (environment), restituisce un booleano ed un numero. Se il booleano è falso la reazione non può avvenire. In caso contrario, invece, avviene. Il numero può invece influenzare la velocità della reazione a seconda della reazione e della distribuzione temporale.
    \item \textbf{Azione} (action): un cambiamento nell’ambiente.
\end{itemize}

\begin{figure}[ht]
    \centering
    \includegraphics[width=.8\linewidth]{figures/alchemistReaction.png}
    \caption{Rappresentazione di una reazione in Alchemist}
    \label{fig:reactionAlchemist}
\end{figure}
\begin{figure}[ht]
    \centering
    \includegraphics[width=.8\linewidth]{figures/alchemistModel.png}
    \caption{Il modello di alchemist}
    \label{fig:rmodelAlchemist}
\end{figure}
\clearpage

\section{Simulazione di riferimento} \label{refSim}
La simulazione utilizzata come riferimento per questo progetto di tesi è presente nella libreria
di modelli di NetLogo\cite{wilensky1997netlogo}. Il modello di riferimento è "Slime"\cite{wilensky1997netlogo}
e per simulare l'aggregazione di tanti singoli organismi in un gruppo si fa riferimento al comportamento delle tartarughe.
Quest'ultime si muovono in uno spazio a griglia e durante il loro movimento rilasciano una particolare molecola
chiamata "feromone" che si deposita in una posizione precisa. L'intero mondo è quindi suddiviso
in tantissime "micro-aree" chiamate "patch". La tartaruga per muoversi 
"annusa" davanti a se, ovverro percepisce se nelle patch vicine è presente del "feromone". Se questo è abbastanza alto, la 
tartaruga si sposterà nella posizione "annusata", in caso contrario la tartaruga si muoverà randomicamente. 
Durante tutto ciò, le "patches" diffonderanno del "feromone" alle varie posizioni vicine e con il passare del tempo 
il "feromone" evaporerà dalla griglia.

\section{Overview su slime mold}
In natura, l'aggregazione delle cellule di muffa mucillaginosa (detti anche funghi mucillaginosi o, in inglese, slime-mold) rappresenta
comportamento in cui entità individuali interagiscono tra di loro per formare strutture complesse e funzionali. 
Lo slime mold, o muffa mucillaginosa è un organismo unicellulare ma, talvolta, può trovarsi anche ad agire come un organismo multicellulare. 
Pur non essendo un fungo è spesso classificato nella stessa categoria per via delle sue caratteristiche affini a quelle di questi organismi.
La particolarità principale della muffa mucillaginosa è che può muoversi come un organismo unicellulare o multicellulare a seconda delle diverse 
condizioni ambientali in cui si trova.
L'habitat principale di questi organismi è il terreno umido, dove di solito si nutrono di foglie morte o di tronchi di alberi in putrefazione.
Quando trova una fonte di cibo, lo slime-mold si aggrega, formando una massa citoplasmatica detta plasmodio, composta da un grand numero cellule. Inoltre,
questa massa può muoversi, "navigando" attraverso il terreno in cerca di cibo.
Quando le risorse alimentari scarseggiano o l'ambiente diventa meno ospitale, lo slime mold può assumere forme diverse: può formare spore, 
resistenti per sopravvivere in condizioni avverse, oppure aggregarsi insieme ad altri individui simili per formare una struttura multicellulare
che si comporta come un'unica entità, condividendone di conseguenza risorse e compiti.

\chapter{Analisi}
\section{Requisiti}
Analizzando il modello presente su NetLogo\cite{wilensky1997netlogo}, possiamo individuare 
le caratteristiche e i requisiti che la simualzione dovrà avere:
\begin{itemize}
    \item Entità "vive", che si muovono e depositino il feromone.
    \item Un ambiente che gestisca la presenza del feromone; in particolare dovrà:
    \begin{itemize}
        \item Permettere il depositarsi della sostanza.
        \item Diffondere la sostanza.
        \item Evaporare la sostanza.
    \end{itemize}
    \item Le entità dovranno avere un concetto di direzione.
    \item Il movimento deve seguire delle regole ben precise.
\end{itemize}
\chapter{Design}

\section{Layer}
Si pensi alla simulazione come se fosse un micro mondo, una "città", complessa
e ricca di informazioni. È di interesse capire il livello di inquinamento di questa città, qualcosa di invisibile
all'occhio umano ma comunque presente nell'ambiente, oppure la temperatura nelle varie aree cittadine. Si ha bisogno
di "inserire" nell'ambiente degli "strati" invisibili che hanno il compito di raccogliere queste informazioni.
È possibile in Alchemist definire questi "strati" di dati, chiamati Layer.
\newline
Nel contesto di questo progetto è stato necessario l'utilizzo di un Layer che avesse la funzione di "rete di raccolta" 
dei feromoni che, nella simulazione di riferimento \ref{refSim}, venivano rilasciati dalle tartarughe nelle varie posizioni
dello spazio. 
Il layer, chiamato \textit{PheromoneLayer} avrà come compiti:
\begin{itemize}
    \item Implementare una struttura dati per tenere traccia della quantità di feromone presente in ogni posizione.
    \item Offrire un modo per aggiornare i valori del feromone.
    \item Condividere con le altre classi la struttura dati contenente la quantità di feromone per posizione.
    \item Lasciare la possibilità all'utente di decidere le dimensioni dell'area totale e di quella di ognuna patch.
\end{itemize}
\subsection{Le posizioni}
Un aspetto di particolare rilevanza è stato il processo decisionale relativo alla gestione delle posizioni collegate 
al deposito del feromone. Nella simulazione di riferimento troviamo uno spazio a griglia, dove l'area totale è suddivisa
in "micro-aree" chiamate "patch". In alchemist non è presente il concetto di "area" o "spazio", necessario per individuare una patch,
in quanto le posizioni sono puntiformi e non necessariamente intere. Il layer implementa un sistema che converte la posizione puntiforme
del Simulatore in una posizione, a sua volta puntiforme, ma che simbolizza l'angolo sinistro inferiore di un quadrato.



\begin{figure}[h!]
    \centering
    \includegraphics[width=.8\linewidth]{figures/pheromoneLayer.jpeg}
    \caption{Struttura PheromoneLayer}
    \label{fig:phLayer}
\end{figure}
%You may also put some code snippet (which is NOT float by default), eg: \cref{lst:random-code}.

%\lstinputlisting[float,language=Java,label={lst:random-code}]{listings/HelloWorld.java}

\chapter{Implementazione}

\chapter{Evaluation}
\chapter{Conclusioni e ringraziamenti}
%----------------------------------------------------------------------------------------
% BIBLIOGRAPHY
%----------------------------------------------------------------------------------------

\backmatter

\nocite{*} % comment this to only show the referenced entries from the .bib file

\bibliographystyle{alpha}
\bibliography{bibliography}

\end{document}