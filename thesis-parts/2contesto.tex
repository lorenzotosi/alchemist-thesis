\chapter{Contesto}
In questo capitolo viene spiegato il contesto scientifico in cui si colloca il lavoro svolto.
%Si inizia con una panoramica sul simulatore NetLogo e sulla simulazione, presente nella sua libreria, utilizzata come riferimento per il progetto di tesi.
%Successivamente viene presentata una panoramica sul fenomeno di aggregazione di singoli organismi in gruppi, con particolare attenzione alla muffa mucillaginosa.
%Infine, viene presentato il simulatore Alchemist e il suo meta-modello.
Nello stato dell'arte attuale sono presenti diversi simulatori che permettono di emulare comportamenti emergenti, come ad esempio \textbf{NetLogo}.
Questo strumento ha rivestito un'importanza fondamentale per un riferimento iniziale e per la comprensione del fenomeno in esame.
In particolare, la \textbf{simulazione di riferimento} ``Slime'' è stata utilizzata come base per lo sviluppo del modello di simulazione in Alchemist.
A partire da ciò, nasce l'interesse di indagare un altro fenomeno dalle caratteristiche estremamente simili: la \textbf{muffa mucillaginosa}.
Questo lavoro di tesi si propone l'obiettivo di sviluppare le principali astrazioni, necessarie per modellare il questo fenomeno,
all'interno del simulatore \textbf{Alchemist}.

\section{NetLogo}

\begin{wrapfigure}{l}{0.25\textwidth}
    \centering
    \includegraphics[width=0.2\textwidth]{figures/net.png}
\end{wrapfigure}

NetLogo\footnote{\url{https://ccl.northwestern.edu/netlogo/}}\space\cite{wilensky1997netlogo} è un ambiente di programmazione e simulazione open-source
progettato per eseguire simulazioni di modelli complessi e dinamici. È stato sviluppato da Uri Wilensky
e collaboratori presso il Center for Connected Learning and Computer-Based Modeling presso la Northwestern University.
Il linguaggio di programmazione di NetLogo è molto semplice e permette di scrivere codice in modo
intuitivo e veloce. Inoltre, NetLogo è dotato di un'interfaccia grafica che permette di visualizzare
in tempo reale il comportamento del sistema simulato.
È costruito sulla base di LOGO, un linguaggio di programmazione educativo sviluppato
da Seymour Papert nel 1967. Questo linguaggio è stato progettato per essere facile da imparare e le entità protagoniste
sono \textit{tartarughe} che l'utente può muovere in uno spazio.

NetLogo è un simulatore ad agenti, ovvero un
programma che simula il comportamento di un insieme di agenti che interagiscono tra di loro e con
l'ambiente circostante. Ogni agente in NetLogo può rappresentare una vasta gamma di entità con propria autonomia
decisionale. Gli agenti più comuni sono le \textit{tartarughe} (riprese da LOGO), ovvero entità ``vive'' che si muovono in uno spazio,
\textit{patch}, ovvero le celle che compongono lo spazio, i \textit{link}, ovvero le connessioni tra le \textit{tartarughe} e 
le \textit{patch} e gli \textit{observer}, ovvero l'agente monitor della simulazione.

\subsection{Simulazione di riferimento}\label{refSim}
\begin{figure}[ht]
    \centering
    \subfigure[]{\includegraphics[width=0.4\textwidth]{figures/net0.png}} 
    \subfigure[]{\includegraphics[width=0.4\textwidth]{figures/netF.png}} 
    \caption{Simulazione di riferimento}\label{fig:refsimfoto}
\end{figure}
La simulazione\footnote{\url{https://www.netlogoweb.org/launch\#http://www.netlogoweb.org/assets/modelslib/Sample\%20Models/Biology/Slime.nlogo}}
utilizzata come riferimento per questo progetto di tesi è presente nella libreria
di modelli di NetLogo. Il modello di riferimento è ``Slime''\space\cite{wilensky1997netlogo}\space\cref{fig:refsimfoto}.
e per simulare l'aggregazione di tanti singoli organismi in un gruppo vengono usati gli agenti sopra descritti, 
ovvero le \textit{patch} e le \textit{tartarughe}.
Quest'ultime si muovono in uno spazio a griglia e durante il loro movimento rilasciano una particolare molecola
chiamata ``feromone'' che si deposita in una posizione precisa. L'intero mondo è quindi suddiviso
in tantissime ``micro-aree'' chiamate \textit{patches}\space\cref{fig:patch}. La \textit{tartaruga} per muoversi 
``annusa'' davanti a sè, ovvero percepisce se nelle \textit{patches} vicine è presente del ``feromone''. Se il valore di quest'ultimo è abbastanza alto, la 
\textit{tartaruga} si sposterà nella posizione ``annusata'', mentre, in caso contrario, la \textit{tartaruga} si muoverà in modo randomico nello spazio circostante. 
Contemporaneamente questo procedimento, le \textit{patches} diffonderanno il ``feromone'' alle posizioni vicine e, con il passare del tempo,
il ``feromone'' tende ad evaporare (ovvero sparire) dalla griglia.

\begin{figure}[ht]
    \centering
    \includegraphics[scale=0.6]{figures/patch.png}
    \caption{Il mondo suddiviso in patch}\label{fig:patch}
\end{figure}

\section{Overview su \textit{slime-mold}}
Nel contesto della presente tesi, sulla base della simulazione di riferimento\ref{refSim} e i comportamenti osservati in essa, si è scelto di studiare un altro 
fenomeno ad essa molto affine: la muffa mucillaginosa.
In natura, l'aggregazione delle cellule di muffa mucillaginosa (detti anche funghi mucillaginosi o, in inglese, \textit{slime-mold}) rappresenta
comportamento in cui entità individuali interagiscono tra di loro per formare strutture complesse e funzionali. 
Lo \textit{slime-mold} è un organismo unicellulare:
pur non essendo un fungo, è spesso classificato nella stessa categoria per via delle sue caratteristiche affini a quelle di questi organismi.
La particolarità principale della muffa mucillaginosa è che può comportarsi come un organismo unicellulare o multicellulare a seconda delle diverse 
condizioni ambientali in cui si trova.
L'habitat principale di questi organismi è il terreno umido, dove di solito si nutrono di foglie morte o di tronchi di alberi in putrefazione.
Quando trova una fonte di cibo, la \textit{slime-mold} si aggrega, formando una massa citoplasmatica detta ``plasmodio'', composta da un grande numero cellule. Inoltre,
questa massa può muoversi, ``navigando'' attraverso il terreno in cerca di cibo.
Infatti, se le risorse alimentari scarseggiano o l'ambiente diventa meno ospitale, lo \textit{slime-mold} può assumere forme diverse: può formare spore, 
resistenti per sopravvivere in condizioni avverse, oppure aggregarsi insieme ad altri individui simili per formare una struttura multicellulare
che si comporta come un'unica entità, condividendone di conseguenza risorse e compiti.

\section{Alchemist}
Alchemist\space\cite{Pianini_2013} è un simulatore DES (Discrete Event System) che estende il modello computazionale 
base delle reazioni chimiche in modo tale da favorirne l'applicabilità a situazioni complesse,
pur mantenendo elevate prestazioni. In particolare, Alchemist si fonda su una versione ottimizzata 
dell'algoritmo di Gillespie\cite{gillespie1977exact} chiamata Next Reaction Method\cite{gibson2000efficient}, esteso in modo tale da poter lavorare 
con un ambiente agile e dinamico dove è possibile aggiungere o rimuovere reazioni, dati e
connessioni topologiche. Le applicazioni già implementate sono varie e comprendono, ad esempio, 
simulazioni di reazioni biochimiche e movimento di pedoni. Il punto di forza del sistema è il 
meta-modello estremamente astratto, la cui effettiva realizzazione è demandata alle ``incarnazioni'',
le quali rappresentano l'implementazione vera e propria delle diverse categorie di simulazioni 
eseguibili all'interno. Attualmente troviamo 4 incarnazioni: 
\begin{itemize}
    \item Protelis
    \item SAPERE
    \item Biochemistry
    \item Scafi
\end{itemize}
\subsection{Il meta-modello} 
Come accennato in precedenza, il meta-modello \cref{fig:alchemist} è uno dei punti di forza maggiori di Alchemist. 
Per meta-modello si intende un tipo di paradigma che descrive la struttura, le regole e le relazioni
che i modelli di dati devono seguire all'interno di un sistema. Esso rappresenta in modo astratto i 
concetti e le relazioni all'interno del dominio di interesse e stabilisce i vincoli e le convenzioni
che tutti i modelli devono usare. Dunque, tutte le incarnazioni sviluppate presentano le stesse entità
``base''. Poichè Alchemist è sviluppato partendo da un'ispirazione orientata alla chimica/biochimica,
le entità presentano nomi riconducibili a quei mondi. Infatti troviamo:
\begin{itemize}
    \item \textbf{Molecole} (\textit{molecule}): il nome di un dato, concettualmente interpretabile come il nome di una variabile.
    \item \textbf{Concentrazioni} (\textit{concentration}): il valore associato alla molecola.
    \item \textbf{Nodi}\label{node} (\textit{node}): il ``contenitore'' di molecole e reazioni.
    \item \textbf{Ambiente} (\textit{environment}): l'astrazione dello spazio; è un “contenitore” di nodi e svolge i seguenti compiti:
    \begin{itemize}
        \item Restituire la posizione dei nodi.
        \item Restituire la distanza tra due nodi.
        \item Supportare il movimento dei nodi, se presente.
    \end{itemize}
    \item \textbf{Vicinato} (\textit{neighborhood}): un entità composta da un nodo centrale e un insieme di nodi vicini.
    \item \textbf{Regola di collegamento} (\textit{linking rule}): una funzione relativa allo stato corrente dell'ambiente che associa ad ogni nodo un vicinato.
    \item \textbf{Reazione} (\textit{reaction}): un qualsiasi evento che provoca un cambiamento dello stato dell'ambiente \cref{fig:alchemist}. È definita da una lista di condizioni, una o più lista di azioni e una distribuzione temporale. La frequenza con la quale avviene una reazione dipende da:
    \begin{itemize}
        \item Un parametro statico \textit{rate}.
        \item Il valore di ogni condizione.
        \item Una \textit{rate equation}, ovvero una equazione che combina il parametro statico (\textit{rate}) con i valori delle condizioni, restituendo un \textit{instantaneous rate}.
        \item Una distribuzione temporale.
    \end{itemize}
    \item \textbf{Condizione} (\textit{condition}): una funzione che, dato lo stato attuale del'ambiente (\textit{environment}), restituisce un booleano ed un numero. Se il booleano è falso la reazione non può avvenire. In caso contrario, invece, avviene. Il numero può invece influenzare la velocità della reazione a seconda della reazione e della distribuzione temporale.
    \item \textbf{Azione} (\textit{action}): un cambiamento nell'ambiente.
\end{itemize}
\begin{figure}[ht]
    \centering
    \subfigure[]{\includegraphics[width=0.49\textwidth]{figures/alchemistReaction.png}} 
    \subfigure[]{\includegraphics[width=0.49\textwidth]{figures/alchemistModel.png}} 
    \caption{(a) Reazione in Alchemist (b) Il modello di alchemist}\label{fig:alchemist}
\end{figure}
\clearpage




