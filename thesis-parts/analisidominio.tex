\chapter{Analisi del dominio}
\section{Dominio}
Il dominio di interesse è caratterizzato dalla presenza di entità in un ambiente.
Queste entità sono vive e si muovono. Durante il loro movimento rilasciano una traccia,
sotto forma di sostanza chimica,
simile ad un odore che può essere percepito dalle altre entità presenti nell'ambiente. Questa sostanza
si deposita in un punto dell'ambiente ed inizia ad espandersi. Con il passare del tempo la sostanza, evaporando, svanisce dall'ambiente.
Quando un'entità, muovendosi, percepisce la presenza di questa sostanza, viene influenzata a muoversi verso la direzione 
in cui l'ha percepita. Questo comportamento, riprodotto in larga scala, porta alla formazione di agglomerati
di entità.
\section{Requisiti}
Analizzando il dominio possiamo individuare e sintetizzare
le caratteristiche e i requisiti che la simualzione dovrà avere:
\begin{itemize}
    \item Entità ``vive'', che si muovono e depositino il feromone.
    \item Un ambiente che gestisca la presenza del feromone; in particolare dovrà:
    \begin{itemize}
        \item Permettere il depositarsi della sostanza.
        \item Diffondere la sostanza.
        \item Evaporare la sostanza.
    \end{itemize}
    \item Le entità dovranno avere un concetto di direzione.
    \item Il movimento deve seguire delle regole ben precise.
\end{itemize}

Non sono stati individuati ulteriori requisiti specifi per la realizzazione di questo progetto di tesi.
Il motivo è che lo scopo di questa ricerca è stato quello di comprendere se il simulatore Alchemist 
potesse essere in grado di simulare questi comportamenti, e in caso affermativo, sviluppare un modello
dimostrativo.