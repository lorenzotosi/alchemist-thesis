\chapter{Conclusioni e lavori futuri}
\section{Conclusioni}
Questa tesi si è posta l'obiettivo di dimostrare come sia possibile simulare comportamenti emergenti in Alchemist sviluppando un modello
di aggregazione di organismi \textit{slime-mold}. 
Lo svolgimento del progetto ha portato con successo lo sviluppo di sorgenti software che permettono di simulare
questo comportamento.
\section{Lavori futuri}
Questa tesi ha dimostrato un modo di poter simulare il comportamento di aggregazione di organismi \textit{slime-mold} in Alchemist:
un possibile sviluppo futuro potrebbe essere quello di simulare altri comportamenti emergenti che sfruttano
i feromoni, proprio come avviene in questo modello.\newline
Inoltre si potrebbe fornire un'interfaccia per poter 
rendere possibile l'apprendimento automatico distribuito del modello sviluppato, come rende possibile la libreria ``PettingZoo''\cite{terry2021pettingzoo}.
Per quanto riguarda invece questo lavoro di tesi, invece, un possibile sviluppo futuro potrebbe consistere nel modificare il modello per permettere all'utente di definire l'angolo di 
scoperta del feromone. Attualmente ogni nodo ``sniffa'' il feromone a 360 gradi, ma sarebbe interessante
poter definire un angolo di visuale, in modo da poter simulare comportamenti più realistici.