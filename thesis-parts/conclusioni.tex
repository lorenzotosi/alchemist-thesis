\chapter{Conclusioni e lavori futuri}
\section{Conclusioni}
Questa tesi si è posta l'obiettivo di dimostrare come sia possibile simulare comportamenti emergenti in Alchemist sviluppando un modello
di aggregazione di organismi \textit{slime-mold}. 
Lo svolgimento del progetto ha portato con successo allo sviluppo di sorgenti software che permettono di simulare
questo comportamento, offrendo inoltre la possibilità di poter variare diversi parametri del modello per poter osservare
come questi influenzino il comportamento. 


\section{Lavori futuri}
Questa tesi ha dimostrato un modo di poter simulare il comportamento di aggregazione di organismi \textit{slime-mold} in Alchemist:
un possibile sviluppo futuro potrebbe essere quello di utilizzare questo modello per simulare altri comportamenti emergenti che avvengono 
grazie al feromone.\newline
Inoltre si potrebbe fornire un'interfaccia per poter 
rendere possibile l'apprendimento automatico distribuito del modello sviluppato.
Sono presenti piattaforme che rendono possibile questo, comead esempio la libreria ``PettingZoo''\cite{terry2021pettingzoo} e sarebbe quindi interessante offrire questa funzionalità.
Infine, un possibile sviluppo futuro di questo lavoro di tesi potrebbe consistere nell'ampliare il modello per permettere all'utente di definire l'angolo di 
scoperta del feromone. Attualmente ogni nodo ``sniffa'' il feromone a 360 gradi, ma sarebbe interessante
poter definire un angolo di visuale, in modo da poter simulare più comportamenti.