\chapter{Implementazione}
\section{Struttura del progetto}
La struttura del progetto, organizzato in package, è la seguente:
\begin{itemize}
    \item \textbf{Layer}: il layer personalizzato della simulazione.
    \item \textbf{Actions}: le azioni della simulazione.
    \item \textbf{GlobalReactions}: le azioni globali della simulazione.
    \item \textbf{NodeProperty}: le proprietà dei nodi della simulazione.
\end{itemize}
Per avviare il progetto in Alchemist, è necessario delineare accuratamente i parametri
e le opzioni desiderate attraverso un file di configurazione YAML\@. Questo file fornisce
le istruzioni necessarie per definire il comportamento del simulatore, specificare
i componenti del sistema e regolare le interazioni tra di essi.
\clearpage
\section{Layer}\label{layer}
\begin{figure}[ht]
    \centering
    \includegraphics[width=.5\linewidth]{figures/pheromoneLayer.png}
    \caption{Struttura PheromoneLayer}\label{fig:phLayer}
\end{figure}
Il \texttt{PheromoneLayer<P extends Position2D<P>>}, layer personalizzato della simulazione, è stato implementato 
come un'interfaccia che estende \texttt{Layer<T, P>} - interfaccia propria di Alchemist -
dove \texttt{T} è il tipo di nodo e \texttt{P} è il tipo di posizione. 
L'utilizzo del parametro \texttt{P} implica che il \texttt{PheromoneLayer} può essere utilizzato con qualsiasi tipo di posizione, ma, nel contesto di questa tesi,
si è preferito sfruttare posizioni \texttt{Position2D<P>} bidimensionali.
Per la sua creazione è necessario definire 5 misure:
\begin{itemize}
    \item \texttt{startX}: la coordinata x di partenza.
    \item \texttt{startY}: la coordinata y di partenza.
    \item \texttt{width}: la larghezza del layer.
    \item \texttt{height}: l'altezza del layer.
    \item \texttt{step}: la dimensione del passo, ovvero la lunghezza del lato di ogni \textit{patch}.
\end{itemize}
Lo \texttt{step} è un parametro fondamentale per la corretta implementazione della simulazione
in quanto Alchemist non possiede il concetto di area, necessaria per individuare una \textit{patch}.
Queste vengono rappresentate come ``aree'' puntiformi, e la loro dimensione (ovvero la distanza di un punto dall'altro) è appunto definita da questo parametro.
Nella simulazione, il nodo deposita il feromone in una qualsiasi posizione, discreta e non obbligatoriamente intera, all'interno dei limiti dello spazio, e il \texttt{PheromoneLayer} 
si occupa di convertire questa posizione in una appartenente ad una \textit{patch}. 
Il \texttt{PheromoneLayer} esegue, quindi, un arrotondamento per eccesso o per difetto, in modo tale da ottenere la posizione della \textit{patch} più vicina.
\begin{figure}[ht]
    \centering
    \includegraphics[width=.5\linewidth]{figures/patch-nodi.png}
    \caption{Rappresentazione grafica delle \textit{patches} puntiformi. Ogni punto rappresenta una patch,
    mentre i quadratini rappresentano i nodi e la freccia indica su che \textit{patch} il nodo depositerà il feromone. In questo esempio
    startX e startY hanno come valore 0, width e height 1 e step 0.5}\label{fig:patch-nodi}
\end{figure}
%devo parlare della Mappa e delle posizioni, di come vengono convertite magari aggiungendo delle foto.

\subsection{Struttura dati}\label{strDati}
Un aspetto di fondamentale importanza riguarda la struttura dati utilizzata per la gestione del feromone.
Per ovviare alla mancanza del concetto di area, è stato utilizzato un \texttt{HashMap<P, Double>} che associa ad ogni posizione \texttt{P} un valore \texttt{Double} di feromone.
Questa mappa viene inizializzata nel costruttore della classe attraverso il metodo \texttt{setupEnviromnent()} che si occupa di popolare la mappa
con tutte le possibili posizioni delle \textit{patch} e di inizializzare il feromone a 0.\newline

\lstinputlisting[language=Java,label={lst:phlayer}]{listings/phlayerSetup.java}

\subsection{Metodi}
I metodi definiti nell'interfaccia e implementati nella classe sono:
\begin{itemize}
    \item \texttt{void evaporate(P position, Double value)}: metodo che permette di far evaporare il feromone. 
    Richiede in input la posizione e il valore del feromone.
    \item \texttt{void deposit(P position, Double value)}: metodo che permette di depositare il feromone.
     Richiede in input la posizione e il valore del feromone.
    \item \texttt{Rectangle getLayerBounds()}: metodo che restituisce un oggetto di tipo \texttt{Rectangle}, contentente i parametri per delineare l'area del Layer.
\end{itemize}
Di grande importanza sono i primi due metodi: \texttt{evaporate} e \texttt{deposit}.
Entrambi sono nominati come le reazioni globali della simulazione e vengono utilizzati dalle stesse per accedere la mappa e modificare il feromone.\newline
\lstinputlisting[language=Java,label={lst:phlayer}]{listings/phLayerDepositAndEvaporate.java}

\section{Actions}
In questa sezione vengono descritte le azioni della simulazione. Possiamo trovare:
\begin{itemize}
    \item \texttt{MoveNode}: azione che permette ad ogni singolo nodo di muoversi.
    \item \texttt{NodeInfo}: azione che permette di controllare le informazioni di ogni singolo nodo.
\end{itemize}

\subsection{MoveNode}\label{moveNode}
La classe \texttt{MoveNode<P extends Position<P> \& Position2D<P>>} incorpora l'intera logica che permette il movimento di ogni singolo nodo. 
Per la sua creazione è necessario che l'utente definisca i seguenti parametri:
\begin{itemize}
    \item \texttt{sniffThreshold}: la soglia di feromone che il nodo deve percepire per potersi muovere.
    \item \texttt{sniffDistance}: la distanza del passo di movimento.
    \item \texttt{wiggleBias}: la tendenza a preferire un movimento casuale oscillatorio in una direzione specifica.
\end{itemize}
Il parametro \texttt{wiggleBias} può essere impostato in 3 modi:
\begin{itemize}
    \item \texttt{0}: il nodo ha il 50\% di muoversi in avanti e il 25\% di muoversi a destra o a sinistra.
    \item \texttt{0 > x <= 40}: il nodo tende a preferire la direzione di sinistra; il valore indica la probabilità di muoversi in quel verso.
    Il valore 40 indica il 100\% di probabilità di muoversi in quella direzione.
    \item \texttt{-40 => x < 0}: il nodo tende a preferire la direzione di destra; il valore indica la probabilità di muoversi in quel verso.
    Il valore -40 indica il 100\% di probabilità di muoversi in quella direzione.
\end{itemize}
La classe \texttt{MoveNode} estende la classe astratta \texttt{AbstractAction<T>}, facente parte del set base di Alchemist, implementandone i metodi astratti.
Tra questi, il metodo \texttt{execute} è il più importante, in quanto si occupa di eseguire l'azione di movimento vera e propria.
La logica di movimento segue questi passi:
\begin{enumerate}
    \item Viene individuata la posizione attuale del nodo.
    \item Questa posizione viene adattata alla \textit{patch} più vicina.
    \item Vengono calcolate le direzioni possibili in base alle patch adiacenti alla posizione calcolata precedentemente che hanno una 
    concentrazione di feromone superiore ad una soglia definita dall'utente (il parametro si chiama \texttt{sniffThreshold}).
    \item Viene identificata la \textit{patch} con la maggiore concentrazione di feromone. Tuttavia, questa, non è sempre individuabile. Vi sono due casi possibili: nel primo, 
        nella \textit{patch} è presente un valore di feromone, ma esso è sotto la soglia minima \texttt{sniffThreshold} e dunque la \textit{patch} viene scartata;
         nel secondo caso, invece, nella \textit{patch} non è presente alcuna traccia di feromone, e dunque questa non viene proprio rilevata.
    \item Se la \textit{patch} è presente:
    \begin{enumerate}
        \item La direzione del nodo viene aggiornata in base alla direzione della \textit{patch} con la maggiore concentrazione di feromone.
        \item Il nodo si muove verso quella \textit{patch} e si aggiorna la direzione del nodo.
    \end{enumerate}
    \item Se la \textit{patch} non è presente:
    \begin{enumerate}
        \item Viene calcolata una direzione casuale tra quelle possibili (dritto, verso destra o verso sinistra a seconda del valore del parametro \texttt{wiggleBias}),
        tenendo in considerazione la direzione attuale del nodo.
        \item Il nodo si muove nella direzione scelta e l'orientamento del nodo viene aggiornato.
    \end{enumerate}
\end{enumerate}

\subsection{NodeInfo}
La classe \texttt{NodeInfo<T, P extends Position<P> \& Position2D<P>>} permette di tracciare le informazioni di ogni singolo nodo. Anche questa
classe estende la classe astratta \texttt{AbstractAction<T>}, implementandone i metodi. Le informazioni osservabili sono le seguenti:
\begin{itemize}
    \item \texttt{PheromoneValue}: la concentrazione di feromone nella \textit{patch} in cui si trova il nodo.
    \item \texttt{direction}: la direzione attuale del nodo.
    \item \texttt{pheromone}: la quantità di feromone che il nodo rilascia.
\end{itemize}

\begin{figure}[ht]
    \centering
    \includegraphics[width=.7\linewidth]{figures/nodeinfo.jpg}
    \caption{Le informazioni del nodo. Si possono osservare nella sezione ``Content''}\label{fig:nodeinfo}
\end{figure}
\clearpage

\section{GlobalReactions}
\begin{figure}[ht]
    \centering
    \includegraphics[width=.7\linewidth]{figures/global.png}
    \caption{Schema delle Global Reaction }\label{fig:global}
\end{figure}
In questa sezione del progetto sono state implementate le reazioni globali della simulazione. Possiamo trovare:
\begin{itemize}
    \item \texttt{Evaporate}: azione che permette di far evaporare il feromone.
    \item \texttt{Deposit}: azione che permette di far diffondere il feromone.
    \item \texttt{Diffuse}: azione che diffonde il feromone nelle \textit{patch} adiacenti.
\end{itemize}
Le classi sopra introdotte estendono la classe astratta \texttt{AbstractGlobalReaction<T, P extends Position<P>}
\& \texttt{Position2D<P>}
la quale implementa l'interfaccia \newline\texttt{PheromoneGlobalReaction<T, P>}. Quest'ultima estende l'interfaccia 
propria di Alchemist \texttt{GlobalReaction<T>} dove sono definite le operazioni da implementare in modo tale che il 
simulatore possa identificare ed utilizzare in modo corretto tutti i sorgenti necessari.
L'interfaccia \texttt{PheromoneGlobalReaction<T, P>} definisce il metodo \texttt{action(PheromoneLayerImpl<P> phLayer)}.
Quest'ultimo è implementato in ogni classe e si occupa di eseguire la logica della reazione.
Si è scelta questa struttura in quanto i metodi definiti dall'interfaccia \texttt{GlobalReaction<T>}
sono i medesimi per ogni azione globale e quindi l'utilizzo della classe astratta \texttt{AbstractGlobalReaction},
che li implementa, permette di evitare la ripetizione inutile del codice.

La simulazione esegue una volta al secondo le reazioni in modo tale da garantire un'evoluzione corretta del feromone nel tempo. Inoltre, le reazioni
in questione vengono eseguite nell'ordine in cui sono state definite nel file di configurazione YAML\@.
\subsection{Evaporate}
Questa classe ha il compito di fare evaporare il feromone dall'ambiente. L'azione di evaporazione è simulata
attraverso la moltiplicazione del valore del feromone per un coefficente di evaporazione compreso tra 0 e 1, 
definito dall'utente. Questo evento ha effetto su ogni singola \textit{patch} in cui è presente il feromone.
Per alterare il valore del feromone di ogni \textit{patch}, viene richiamato il metodo 
\texttt{evaporate} del \texttt{PheromoneLayer}\space\ref{layer}.
\newline
\lstinputlisting[language=Java,label={lst:evaporate}]{listings/evaporate.java}

\subsection{Deposit}
Il sorgente protagonista di questa sotto-sezione compie l'azione di depositare il feromone. Dall'ambiente
vengono individuate le posizioni di tutti i nodi e viene poi richiamato il metodo \texttt{deposit} del \texttt{PheromoneLayer}\space\ref{layer}
per modificare il valore del feromone, associato alla posizione attuale del nodo, presente nella struttura dati\space\ref{strDati}.\newline
\lstinputlisting[language=Java,label={lst:deposit}]{listings/deposit.java}

\subsection{Diffuse}
Quest'ultima classe si occupa di diffondere il feromone nelle \textit{patch} adiacenti. Vengono quindi individuate tutte
le \textit{patch}, e per ognuna si calcola il suo vicinato. Se la \textit{patch} ha un valore di feromone superiore ad una soglia definita dall'utente,
si procede a diffondere il feromone. La diffusione è emulata attraverso la moltiplicazione del valore del feromone
per un coefficente di diffusione deciso dall'utente. Questa reazione, concettualmente, è simile alla \texttt{Deposit} in quanto
nel vicinato di una \textit{patch} viene depositato del feromone, e per questo motivo, viene richiamato il metodo \texttt{deposit} del \texttt{PheromoneLayer}\space\ref{layer}.
\newline
\lstinputlisting[language=Java,label={lst:diffuse}]{listings/diffuse.java}

\section{NodeProperty}
La classe \texttt{DirectionProperty} è stata implementata per poter associare ad ogni nodo una proprietà che rispecchiasse la sua direzione attuale.
Estende la classe \texttt{AbstractNodeProperty<T>} fornita dal set base di Alchemist.
L'enum \texttt{Direction}, che definisce le direzioni possibili in cui un nodo può muoversi, è propedeutico all'utilizzo di questa classe.
L'enum definisce i 4 punti cardinali e i loro punti intermedi. Infatti, un nodo può muoversi in 8 direzioni diverse e, tenere traccia di
queste, è fondamentale per la corretta esecuzione della simulazione. L'enum, oltre a contenere le definizioni delle direzioni, contiene anche,
per ognuna di esse, metodi che ne ritornano le coordinate x e y e le direzioni adiacenti. All'avvio del programma, ad ogni nodo 
viene assegnata una direzione in modo randomico. 

La \texttt{NodeProperty} è una proprietà fondamentale per il corretto funzionamento dell'azione \texttt{MoveNode}
\space\ref{moveNode}; senza di essa, infatti, il movimento del nodo risulterebbe irrealistico in quanto si muoverebbe in modo completamente casuale.
Con l'implementazione di questa classe e una logica di movimento (implementata nella classe \texttt{MoveNode}\space\ref{moveNode})
pensata per sfruttare questa componente, invece,
si osserva che il movimento di un singolo nodo risulta piuttosto ``armonioso'' e realistico, più simile a quello di un essere vivente che 
esplora l'ambiente circostante.

\section{Simulazioni}
In questa sezione vengono presentati i risultati delle simulazioni effettuate e i loro valori.
\subsection{Simulazione 1}\label{sim1}
I valori di questa simulazione\space \cref{fig:sim1} sono:
\begin{itemize}
    \item Numero di nodi: 500
    \item \texttt{sniffThreshold}: 1.5
    \item \texttt{wiggleBias}: 0
    \item \texttt{evaporation}: 0.6
    \item \texttt{diffusion}: 0.5
    \item \texttt{deposit}: 1
    \item \texttt{startX}: -15
    \item \texttt{startY}: -15
    \item \texttt{width}: 30
    \item \texttt{height}: 30
    \item \texttt{step}: 0.5
    \item \texttt{customDiffusionTreshold}: 5
\end{itemize}
\begin{figure}[p]
    \centering
    \subfigure[]{\includegraphics[width=0.32\textwidth]{figures/rect0.png}} 
    \subfigure[]{\includegraphics[width=0.32\textwidth]{figures/rect100.png}} 
    \subfigure[]{\includegraphics[width=0.32\textwidth]{figures/rectFine.png}}
    \caption{(a) Inizio (b) Dopo 100 secondi (c) Dopo 300 secondi}\label{fig:sim1}
\end{figure}

\subsection{Simulazione 2}\label{sim2}
I valori di questa simulazione\space \cref{fig:sim2} sono:
\begin{itemize}
    \item Numero di nodi: 500
    \item \texttt{sniffThreshold}: 4
    \item \texttt{wiggleBias}: 0
    \item \texttt{evaporation}: 0.6
    \item \texttt{diffusion}: $\frac{1}{18}$
    \item \texttt{deposit}: 1
    \item \texttt{startX}: -15
    \item \texttt{startY}: -15
    \item \texttt{width}: 30
    \item \texttt{height}: 30
    \item \texttt{step}: 0.5
    \item \texttt{customDiffusionTreshold}: 1
\end{itemize}
\begin{figure}[p]
    \centering
    \subfigure[]{\includegraphics[width=0.32\textwidth]{figures/slow0.png}} 
    \subfigure[]{\includegraphics[width=0.32\textwidth]{figures/slow100.png}} 
    \subfigure[]{\includegraphics[width=0.32\textwidth]{figures/slowF.png}}
    \caption{(a) Inizio (b) Dopo 100 secondi (c) Dopo 300 secondi}\label{fig:sim2}
\end{figure}

\subsection{Simulazione 3}\label{sim3}
I valori di questa simulazione\space \cref{fig:sim3} sono:
\begin{itemize}
    \item Numero di nodi: 100
    \item \texttt{sniffThreshold}: 1.5
    \item \texttt{wiggleBias}: 0
    \item \texttt{evaporation}: 0.6
    \item \texttt{diffusion}: 0.5
    \item \texttt{deposit}: 1
    \item \texttt{startX}: -15
    \item \texttt{startY}: -15
    \item \texttt{width}: 30
    \item \texttt{height}: 30
    \item \texttt{step}: 0.5
    \item \texttt{customDiffusionTreshold}: 5
\end{itemize}
\begin{figure}[p]
    \centering
    \subfigure[]{\includegraphics[width=0.32\textwidth]{figures/cento0.png}} 
    \subfigure[]{\includegraphics[width=0.32\textwidth]{figures/cento100.png}} 
    \subfigure[]{\includegraphics[width=0.32\textwidth]{figures/centoF.png}}
    \caption{(a) Inizio (b) Dopo 100 secondi (c) Dopo 300 secondi}\label{fig:sim3}
\end{figure}

\subsection{Simulazione 4}\label{sim4}
I valori di questa simulazione\space \cref{fig:sim4} sono:
\begin{itemize}
    \item Numero di nodi: 100
    \item \texttt{sniffThreshold}: 4
    \item \texttt{wiggleBias}: 0
    \item \texttt{evaporation}: 0.6
    \item \texttt{diffusion}: $\frac{1}{18}$
    \item \texttt{deposit}: 1
    \item \texttt{startX}: -15
    \item \texttt{startY}: -15
    \item \texttt{width}: 30
    \item \texttt{height}: 30
    \item \texttt{step}: 0.5
    \item \texttt{customDiffusionTreshold}: 1
\end{itemize}
\begin{figure}[p]
    \centering
    \subfigure[]{\includegraphics[width=0.32\textwidth]{figures/centoNo0.png}} 
    \subfigure[]{\includegraphics[width=0.32\textwidth]{figures/centoNo100.png}} 
    \subfigure[]{\includegraphics[width=0.32\textwidth]{figures/centoNoF.png}}
    \caption{(a) Inizio (b) Dopo 100 secondi (c) Dopo 300 secondi}\label{fig:sim4}
\end{figure}

\subsection{Simulazione 5}\label{sim5}
I valori di questa simulazione\space \cref{fig:sim5} sono:
\begin{itemize}
    \item Numero di nodi: 500
    \item \texttt{sniffThreshold}: 4
    \item \texttt{wiggleBias}: 0
    \item \texttt{evaporation}: 0.5
    \item \texttt{diffusion}: 1
    \item \texttt{deposit}: 2
    \item \texttt{startX}: -10
    \item \texttt{startY}: -10
    \item \texttt{width}: 20
    \item \texttt{height}: 20
    \item \texttt{step}: 0.5
    \item \texttt{customDiffusionTreshold}: 10
\end{itemize}
\begin{figure}[p]
    \centering
    \subfigure[]{\includegraphics[width=0.32\textwidth]{figures/small0.png}} 
    \subfigure[]{\includegraphics[width=0.32\textwidth]{figures/small100.png}} 
    \subfigure[]{\includegraphics[width=0.32\textwidth]{figures/smallF.png}}
    \caption{(a) Inizio (b) Dopo 100 secondi (c) Dopo 300 secondi}\label{fig:sim5}
\end{figure}

\subsection{Risultati}
I risultati che si possono osservare dalle simulazioni eseguite sono i seguenti:
le simulazioni 1\space\ref{sim1} e 2\space\ref{sim2}, che presentano un numero di nodi pari a 500, mostrano un andamento simile.
La loro differenza principale risiede nel valore di \texttt{diffusion}. Possiamo notare come, dopo 100 secondi,
in entrambe le simulazioni si osserva l'aggregazione, in più punti, dei nodi. La prima simulazione 
mostra un'aggregazione più ``densa'' rispetto alla seconda, in quanto sono presenti meno nodi ``liberi'' nello spazio.
Dopo 300 secondi la prima simulazione mostra che ogni nodo si è aggregato, mentre la seconda presenta ancora nodi vaganti.

Se confrontiamo le simulazioni 3\space\ref{sim3} e 4\space\ref{sim4}, che presentano gli stessi valori delle precedenti, ma con un numero di nodi pari a 100, possiamo osservare come, dopo 100 secondi,
la terza simulazione inizia a sviluppare delle aggregazioni, che vengono confermate e rafforzate dopo 300 secondi.
La quarta simulazione, invece non presenta alcun fenomeno di questo tipo.

Infine, la simulazione 5\space\ref{sim5}, che presenta un ambiente di dimensioni inferiori rispetto alle altre e 
valori più alti, mostra che l'aggregazione dei nodi è più rapida.