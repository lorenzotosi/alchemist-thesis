\chapter{Implementazione}
\section{Struttura del progetto}
La struttura del progetto, organizzato in package, è la seguente:
\begin{itemize}
    \item \textbf{Layer}: il layer personalizzato della simulazione.
    \item \textbf{Actions}: le azioni della simulazione.
    \item \textbf{GlobalReactions}: le azioni globali della simulazione.
    \item \textbf{NodeProperty}: le proprietà dei nodi della simulazione.
\end{itemize}
Per avviare il progetto in Alchemist, è necessario configurare accuratamente i parametri
e le opzioni desiderate attraverso un file di configurazione YAML\@. Questo file fornisce
le istruzioni necessarie per definire il comportamento del simulatore, specificare
i componenti del sistema e regolare le interazioni tra di essi.
\clearpage
\section{Layer}
\begin{figure}[ht]
    \centering
    \includegraphics[width=.5\linewidth]{figures/pheromoneLayer.png}
    \caption{Struttura PheromoneLayer}\label{fig:phLayer}
\end{figure}
Il \texttt{PheromoneLayer<P extends Position2D<P>>}, layer personalizzato della simulazione, è stato implementato 
come una interfaccia che estende \texttt{Layer<T, P>}, dove \texttt{T} è il tipo di nodo e \texttt{P} è il tipo di posizione,
interfaccia propria di Alchemist. L'utilizzo del parametro \texttt{P} implica che il \texttt{PheromoneLayer} può essere utilizzato con qualsiasi tipo di posizione, ma
è stato pensato per sfruttare le posizioni \texttt{Position2D<P>} bidimensionali.
Per la sua creazione è necessario definire 5 misure:
\begin{itemize}
    \item \texttt{startX}: la coordinata x di partenza.
    \item \texttt{startY}: la coordinata y di partenza.
    \item \texttt{width}: la larghezza del layer.
    \item \texttt{height}: l'altezza del layer.
    \item \texttt{step}: la dimensione del passo, ovvero la lunghezza del lato di ogni \textit{patch}.
\end{itemize}
Lo \texttt{step} è un parametro fondamentale per la corretta implementazione della simulazione
in quanto Alchemist non possiede il concetto di area, necessaria per individuare una \textit{patch}.
Queste vengono rappresentate come ``aree'' puntiformi, e la loro dimensione (ovvero la distanza di un punto dall'altro) è appunto definita da questo parametro.
Nella simulazione, il nodo deposita il feromone in una qualsiasi posizione, discreta e non obbligatoriamente intera, all'interno dei limiti dello spazio, e il \texttt{PheromoneLayer} 
si occupa di convertire questa posizione in una appartenente ad una \textit{patch}. 
Esegue quindi un arrotondamento per eccesso o per difetto, in modo tale da ottenere la posizione della \textit{patch} più vicina.
\begin{figure}[ht]
    \centering
    \includegraphics[width=.5\linewidth]{figures/patch-nodi.png}
    \caption{Rappresentazione grafica delle \textit{patches} puntiformi. Ogni punto rappresenta una patch,
    mentre i quadratini rappresentano i nodi e la freccia indica su che \textit{patch} il nodo depositerà il feromone. In questo esempio
    startX e startY hanno come valore 0, width e height 1 e step 0.5}\label{fig:patch-nodi}
\end{figure}
%devo parlare della Mappa e delle posizioni, di come vengono convertite magari aggiungendo delle foto.
\clearpage
\subsection{Struttura dati}
Un aspetto di fondamentale importanza riguarda la struttura dati utilizzata per la gestione del feromone.
Per ovviare alla mancanza del concetto di area, è stato utilizzato un \texttt{HashMap<P, Double>} che associa ad ogni posizione \texttt{P} un valore \texttt{Double} di feromone.
Questa mappa viene inizializzata nel costruttore della classe attraverso il metodo \texttt{setupEnviromnent()} che si occupa di popolare la mappa
con tutte le possibili posizioni delle \textit{patch} e di inizializzare il feromone a 0.
\lstinputlisting[language=Java,label={lst:phlayer}]{listings/phlayersetup.java}
\subsection{Metodi}
I metodi definiti nell'interfaccia e implementati nella classe sono:
\begin{itemize}
    \item \texttt{void evaporate(P position, Double value)}: metodo che permette di far evaporare il feromone. 
    Richiede in input la posizione e il valore del feromone.
    \item \texttt{void deposit(P position, Double value)}: metodo che permette di far diffondere il feromone.
     Richiede in input la posizione e il valore del feromone.
    \item \texttt{Rectangle getLayerBounds()}: metodo che restituisce un oggetto di tipo \texttt{Rectangle}.
\end{itemize}
Di rilevante importanza sono i primi due metodi: \texttt{evaporate} e \texttt{deposit}.
Entrambi sono nominati come le reazioni globali della simulazione e vengono utilizzati dalle stesse per accedere la mappa e modificare il feromone.
\lstinputlisting[language=Java,label={lst:phlayer}]{listings/layer.java}

\section{Actions}
In questa sezione verranno descritte le azioni della simulazione. Possiamo trovare:
\begin{itemize}
    \item \texttt{MoveNode}: azione che permette di far muovere ogni singolo nodo.
    \item \texttt{NodeInfo}: azione che permette di osservare le informazioni di ogni singolo nodo.
\end{itemize}

\subsection{MoveNode}
La classe \texttt{MoveNode<P extends Position<P> \& Position2D<P>>} incorpora l'intera logica che permette il movimento di ogni singolo nodo. 
Per la sua creazione è necessario che l'utente definisca i seguenti parametri:
\begin{itemize}
    \item \texttt{sniffThreshold}: la soglia di feromone che il nodo deve percepire per potersi muovere.
    \item \texttt{sniffDistance}: la distanza del passo di movimento.
    \item \texttt{wiggleBias}: la tendenza a preferire un movimento casuale oscillatorio in una direzione specifica.
\end{itemize}
Il parametro \texttt{wiggleBias} può essere deciso in 3 modi:
\begin{itemize}
    \item \texttt{0}: il nodo ha il 50\% di muoversi in avanti e il 25\% di muoversi a destra o sinistra.
    \item \texttt{0 > x <=40}: il nodo tende a preferire la direzione di sinistra; il valore indica la probabilità di muoversi in quel verso.
     40 indica il 100\% di probabilità di muoversi in quella direzione.
    \item \texttt{-40 > x < 0}: il nodo tende a preferire la direzione di destra; il valore indica la probabilità di muoversi in quel verso.
    -40 indica il 100\% di probabilità di muoversi in quella direzione.
\end{itemize}
Questa classe estende la classe astratta \texttt{AbstractAction<T>}, originale di Alchemist, implementandone i metodi astratti.
Tra questi, il metodo \texttt{execute} è il più importante, in quanto si occupa di eseguire l'azione vera e propria.
La logica di movimento segue questi passi:
\begin{enumerate}
    \item Viene individuata la posizione attuale del nodo.
    \item Questa posizione viene adattata alla \textit{patch} più vicina.
    \item Vengono calcolate le direzioni possibili in base alle patch adiacenti alla posizione calcolata precedentemente che hanno una 
    concentrazione di feromone superiore a una soglia definita dall'utente
    (\texttt{sniffThreshold}).
    \item Viene identificata la \textit{patch} con la maggiore concentrazione di feromone. Questa però non è sempre presente;
    \item Se la \textit{patch} è presente:
    \begin{enumerate}
        \item La direzione del nodo viene aggiornata in base alla direzione della \textit{patch} con la maggiore concentrazione di feromone.
        \item Il nodo si muove in quella direzione e si aggiorna la direzione del nodo.
    \end{enumerate}
    \item Se la \textit{patch} non è presente:
    \begin{enumerate}
        \item Viene calcolata una direzione casuale tra quelle possibili tenendo in considerazione la direzione attuale del nodo.
        \item Il nodo si muove in quella direzione e si aggiorna la direzione del nodo.
    \end{enumerate}
\end{enumerate}

\subsection{NodeInfo}
La classe \texttt{NodeInfo<T, P extends Position<P> \& Position2D<P>>} permette di osservare le informazioni di ogni singolo nodo. Anche questa
classe estende la classe astratta \texttt{AbstractAction<T>}, implementandone i metodi astratti. Le informazioni osservabili sono le seguenti:
\begin{itemize}
    \item \texttt{PheromoneValue}: la concentrazione di feromone nella \textit{patch} in cui si trova il nodo.
    \item \texttt{direction}: la direzione attuale del nodo.
    \item \texttt{pheromone}: la quantità di feromone che il nodo rilascia.
\end{itemize}

\begin{figure}[ht]
    \centering
    \includegraphics[width=.7\linewidth]{figures/nodeinfo.jpg}
    \caption{Le informazioni del nodo. Si possono osservare nella sezione ``Content''}\label{fig:nodeinfo}
\end{figure}
\clearpage

\section{GlobalReactions}
\begin{figure}[ht]
    \centering
    \includegraphics[width=.7\linewidth]{figures/global.png}
    \caption{Schema delle Global Reaction }\label{fig:global}
\end{figure}
In questa sezione verranno descritte le azioni globali della simulazione. Possiamo trovare:
\begin{itemize}
    \item \texttt{Evaporate}: azione che permette di far evaporare il feromone.
    \item \texttt{Deposit}: azione che permette di far diffondere il feromone.
    \item \texttt{Diffuse}: azione che diffonde il feromone nelle patch adiacenti.
\end{itemize}
