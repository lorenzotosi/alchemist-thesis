\chapter{Introduzione}\label{chap:introduzione}

Nel vasto campo della ricerca scientifica, negli ultimi anni i comportamenti complessi emergenti sono oggetto di 
crescente interesse e studio. Questi fenomeni rappresentano il manifestarsi di comportamenti collettivi che sorgono
dall'interazione dinamica di molteplici componenti di un sistema, difficilmente prevedibili se si considerano solamente 
le leggi che regolano il comportamento del singolo. 

In natura questa caratteristica comportamentale è osservabile in un grandissimo numero di ambiti: si pensi, ad esempio, al regno 
animale, dove è possibile ritrovare speciali ``forme'' e comportamenti di stormi di uccelli oppure di banchi di pesci; lo stesso accade 
agli esseri umani in contesti come il traffico cittadino, il mercato della borsa valori o il gioco del poker.

Un esempio significativo di comportamento emergente è quello osservabile in biologia in una colonia di formiche. Nonostante le formiche, se considerate 
come esseri ``singoli'' seguano regole di comportamento molto semplici,
l'interazione tra di esse dà origine ad una ``comunità'' omogenea e, seppure sia assente una struttura gerarchica, sono presenti una serie di modelli condivisi
complessi per quanto riguarda la ricerca del cibo, la costruzione di nidi e la difesa del territorio.
Ogni formica reagisce a degli stimoli, ovvero tracce chimiche provenienti da altre formiche e, al contempo, essa stessa lascia segnali agli altri membri
della comunità: si crea così una reazione a catena che coinvolge tutte le formiche della colonia, che tendono a imitare il comportamento delle altre.
Questo fenomeno è simile ad altre strutture emergenti presenti in natura e riscontrate sia negli ``insetti sociali'' (e.g.\ termiti, vespe, api,\dots),
ovvero insetti che formano colonie con mansioni diversificate, sia, in generale, in animali che vivono in gruppo 
(come pesci, tartarughe, mandrie di mammiferi,\ldots). Questa tipologia di eventi, generalmente, si basa principalmente su feromoni o odori chimici.

%Perché è interessante la simulazione di questi comportamenti?
Nel contesto scientifico, simulare in un ambiente protetto questo tipo di fenomeni è estremamente importante per diversi motivi:
\begin{itemize}
    \item Comprenderne la complessità: i fenomeni complessi, come detto sopra, sono caratterizzati da interazioni dinamiche tra
    i componenti del sistema di riferimento. La simulazione diventa una risorsa chiave per esplorare, studiare e comprendere 
    moltissimi aspetti di queste dinamiche e permette di osservare le interazioni dei diversi elementi in infiniti modi.
    \item Prevedere il comportamento del sistema: poiché questi fenomeni sono altamente aleatori, la simulazione può essere 
    eseguita per cercare di prevedere e avere maggior consapevolezza del comportamento
    futuro di un sistema emergente in modo tale da poter prendere delle decisioni informate.   
\end{itemize}

L'obiettivo di questa tesi è esplorare il fenomeno dell'aggregazione di questi organismi, sviluppando un sistema software che 
si interfacci ed utilizzi a pieno tutti gli elementi chiave del simulatore Alchemist. Quest'ultimo, infatti, permette di riprodurre eventi appartenenti 
a domini estremamente differenti tra loro, come simulazioni chimiche o il comportamento di pedoni in diverse situazioni.
\clearpage
%\paragraph{Structure of the Thesis}

La tesi presenta la seguente struttura:
\begin{itemize}
    \item \textbf{Contesto}: in questo capitolo si introduce il contesto scientifico in cui si colloca il lavoro svolto,
    presentando le tecnologie adottate e la simulazione di riferimento.
    \item \textbf{Analisi}: in questo capitolo si analizzano i requisiti e le specifiche del sistema da sviluppare.
    \item \textbf{Design}: in questo capitolo si illustra il design del sistema, presentando le scelte progettuali e le motivazioni che hanno portato a queste.
    \item \textbf{Implementazione}: in questo capitolo si illustra l'implementazione del sistema, presentando le scelte implementative e mostrando parti di codice significative. Infine si discutono i risultati ottenuti dalle simulazioni.
    \item \textbf{Conclusioni e sviluppi futuri}: in questo capitolo si presentano le conclusioni del lavoro svolto e si discutono possibili sviluppi futuri.
\end{itemize}